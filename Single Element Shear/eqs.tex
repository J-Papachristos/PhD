\documentclass{article}
\usepackage{graphicx}
\usepackage{multirow}
\usepackage{amsmath,amssymb,amsfonts}
\usepackage{amsthm}
\usepackage{geometry}
 \geometry{
 a4paper,
 total={170mm,257mm},
 left=20mm,
 top=10mm,
 }


\begin{document} % chktex-file 44

\title{Geometrically Nonlinear Finite Element Analysis --- Stiffness Equations}
\date{}
\maketitle

\noindent The Linear Stiffness Matrix Equation is written as:
\begin{align}
    \left[
        \begin{array}{c|c}
            K_{\text{ff}} & K_{\text{fs}} \\
            \hline
            K_{\text{sf}} & K_{\text{ss}} \\
        \end{array}
        \right]
    \cdot
    \left[
        \begin{array}{c}
            u_{\text{f}} \\
            \hline
            u_{\text{s}} \\
        \end{array}
        \right]
    = \left[
        \begin{array}{c}
            0 \\
            \hline
            0 \\
        \end{array}
        \right]
    \nonumber
\end{align}
Solving for the Free DOFs vector $\vec{u_{\text{f}}}$, gives:
\begin{align}
    \left[K_{\text{ff}}\right]\cdot \vec{u_{\text{f}}}
    + \left[K_{\text{fs}}\right]\cdot \vec{u_{\text{s}}} = 0
    \nonumber
\end{align}
Where $\vec{u_{\text{s}}}$ is the known vector of Fixed DOFs (Boundary Conditions). Therefore, the known part of the Equation is moved to the RHS and the Equation is rewritten as:
\begin{align}
    \left[K_{\text{ff}}\right]\cdot \vec{u_{\text{f}}} = \vec{b_{\text{f}}}
    \nonumber
\end{align}
The Non-Linear Stiffness Equation for the $i$-th timestep is written as:
\begin{align}
    \left[K_{\text{T}}\right]\cdot \Delta\vec{u_{\text{f}}} = \vec{R}^i = \vec{f_{\text{ext}}} - \vec{f_{\text{int}}}^i
    \nonumber
\end{align}
$K_{\text{T}}$ is the Tangent Stiffness Matrix, and is defined as $K_{\text{T}}=K_{\text{L}}+K_{\text{NL}}$. $K_{\text{L}}$ is the Linear Stiffness Matrix $K_{\text{ff}}$, as previously defined, whereas $K_{\text{NL}}$ is the Non-Linear Stiffness Matrix. Both matrices are calculated on a per-element basis, as a 24-by-24 Local Stiffness Matrix as follows:

\begin{align}
     & {\left[K_{\text{L}}\right]}_{24\times24}
    =\int_{V}{
        {\left[B_\text{L}^T\right]}_{24\times6}
            {\left[D\right]}_{6\times6}
            {\left[B_\text{L}\right]}_{6\times24}
        \,dV}
    \nonumber                                    \\
     & {\left[K_{\text{NL}}\right]}_{24\times24}
    =\int_{V}{
    {\left[B_\text{NL}^T\right]}_{24\times9}
        {\left[S\right]}^i_{9\times9}
    {\left[B_\text{NL}\right]}_{9\times24}
    \,dV}
    \nonumber
\end{align}
Where $\left[D\right]$ is the Elasticity Matrix, $S^i$ is the 2nd Piola-Kirchhoff Stress Matrix, defined as:
\begin{align}
    \left[
        \begin{matrix}
            \tau_{xx} & \tau_{xy} & \tau_{xz} & 0         & 0         & 0         & 0         & 0         & 0         \\
            \tau_{xy} & \tau_{yy} & \tau_{yz} & 0         & 0         & 0         & 0         & 0         & 0         \\
            \tau_{xz} & \tau_{yz} & \tau_{zz} & 0         & 0         & 0         & 0         & 0         & 0         \\
            0         & 0         & 0         & \tau_{xx} & \tau_{xy} & \tau_{xz} & 0         & 0         & 0         \\
            0         & 0         & 0         & \tau_{xy} & \tau_{yy} & \tau_{yz} & 0         & 0         & 0         \\
            0         & 0         & 0         & \tau_{xz} & \tau_{yz} & \tau_{zz} & 0         & 0         & 0         \\
            0         & 0         & 0         & 0         & 0         & 0         & \tau_{xx} & \tau_{xy} & \tau_{xz} \\
            0         & 0         & 0         & 0         & 0         & 0         & \tau_{xy} & \tau_{yy} & \tau_{yz} \\
            0         & 0         & 0         & 0         & 0         & 0         & \tau_{xz} & \tau_{yz} & \tau_{zz} \\
        \end{matrix}
        \right]
    \nonumber
\end{align}
Where $\tau_{ij}$ are the 2nd Piola Stresses.
The Linear Deformation Matrix $\left[B_{\text{L}}\right]$ and the Non-Linear Deformaiton Matrix $\left[B_{\text{NL}}\right]$ are calculated as:
\begin{align}
    \left[B_{\text{L}}\right]
    =\left[
        \begin{array}{ccc|c}
            \dfrac{\partial{N}}{\partial{x}} & 0                                & 0                                & \cdots \\
            0                                & \dfrac{\partial{N}}{\partial{y}} & 0                                & \cdots \\
            0                                & 0                                & \dfrac{\partial{N}}{\partial{z}} & \cdots \\
            \dfrac{\partial{N}}{\partial{y}} & \dfrac{\partial{N}}{\partial{x}} & 0                                & \cdots \\
            0                                & \dfrac{\partial{N}}{\partial{z}} & \dfrac{\partial{N}}{\partial{y}} & \cdots \\
            \dfrac{\partial{N}}{\partial{z}} & 0                                & \dfrac{\partial{N}}{\partial{z}} & \cdots \\
        \end{array}
        \right]
    ,\ \left[B_{\text{NL}}\right]
    =\left[
        \begin{array}{ccc|c}
            \dfrac{\partial{N}}{\partial{x}} & 0                                & 0                                & \cdots \\
            \dfrac{\partial{N}}{\partial{y}} & 0                                & 0                                & \cdots \\
            \dfrac{\partial{N}}{\partial{z}} & 0                                & 0                                & \cdots \\
            0                                & \dfrac{\partial{N}}{\partial{x}} & 0                                & \cdots \\
            0                                & \dfrac{\partial{N}}{\partial{y}} & 0                                & \cdots \\
            0                                & \dfrac{\partial{N}}{\partial{z}} & 0                                & \cdots \\
            0                                & 0                                & \dfrac{\partial{N}}{\partial{x}} & \cdots \\
            0                                & 0                                & \dfrac{\partial{N}}{\partial{y}} & \cdots \\
            0                                & 0                                & \dfrac{\partial{N}}{\partial{z}} & \cdots \\
        \end{array}
        \right]
    \nonumber
\end{align}

\noindent The vector $\vec{f_{\text{ext}}}$ is equal to the vector $\vec{b_{f}}$ found in the Linear Stiffness Equation. The vector $\vec{f_{\text{int}}}$ is calculated per timestep as:

\begin{align}
     & \vec{f_{\text{int}}}_{24\times6}
    =\int_{V}{
    {\left[B_\text{L}^T\right]}_{24\times6}
    \vec{S_{\text{v}}}^i_{6\times1}
    \,dV}
    \nonumber
\end{align}
Where $\vec{S_{\text{v}}}^i_{6\times1}$ is The 2nd Piola Stress Vector (Voigt Notation).

\noindent After solving the Nonlinear Equation by iterating, the displacement vector $\vec{u_{\text{f}}}$ is updated as $\vec{u_{\text{f}}}^{i+1} = \Delta\vec{u_{\text{f}}}^{i} + \vec{u_{\text{f}}}^{i}$.

\end{document}